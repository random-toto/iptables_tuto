\documentclass[a4paper,11pt]{report}
\usepackage[utf8]{inputenc}
\usepackage[T1]{fontenc}
\usepackage[francais]{babel}

\title{Tutoriel iptables v. 1.2.2}
\author{Oskar Andreason} 
\date{1 janvier 2006}

\begin{document}

\maketitle 

\tableofcontents 

\part{Tutoriel} % Partie I

% --------------------------------------------------------- Chapitre 1

\chapter{Introduction} % Chapitre 1

\section{But du document} % I - 1 - 1
Eh bien il manquait des informations à propos d'iptables et du parefeu netfilter dans les noyaux linux 2.4.x. Entre autres je vais essayer de répondre aux questions liés aux machines à état. La plupart sera illustrée dans l'exemple $ rc.firewall.txt $ que vous pouvez utiliser dans vos scripts dand $ /etc/rc.d/ $. Et oui, ce texte provient du HoxTo décrivant le masquage d'IP, pour ceux qui l'ont reconnu.

Il y a aussi un petit script que j'ai écrit pour la remise à zéro en cas d'erreur grave, sous le nom $rc.flush-iptables.txt $.

\section{Comment ce document a été écrit} % I - 1 - 2

\section{Termes utilisés dnas ce document} % I - 1 - 3

\section{Et ensuite ?} % I - 1 - 4

% --------------------------------------------------------- Chapitre 2

\chapter{retour sur la pile TCP/IP} % I - 2

\section{Les couches de TCP/IP} % I - 2 - 1

\section{Caractéristiques d'IP}

\section{En-têtes d'IP}

\section{Caractéristiques de TCP}

\section{En-têtes de TCP}

\section{Caractéristiques de UDP}

\section{En-têtes de UDP}

\section{Caractéristiques de ICMP}

\section{En-têtes de ICMP}

\subsection{La requête ICMP et sa réponse}

\subsection{ICMP : destination injoignable}

\subsection{ICMP : Extinction de la source (source quench)}

\subsection{Redirection}

\subsection{TTL vaut 0}

\subsection{Problème de paramètre}

\subsection{Requête d'heure et réponse}

\subsection{Demande de masque sous-réseau et réponse}

%
\section{Caractéristiques SCTP}

\subsection{Initialisation et association}

\subsection{envoi de données et contrôle de session}

\subsection{Extinction et Avortement de session}

%
\section{En-tête SCTP}

\subsection{Format générique du header}

\subsection{Headers communs et génériques}

\subsection{Le segment SCTP ABORT}

\subsection{Le segment SCTP COOKIE ACK}

\subsection{Le segment SCTP COOKIE ECHO}

\subsection{Le segment SCTP DATA}

\subsection{Le segment SCTP ERROR}

\subsection{Le segment SCTP HEARTBEAT}

\subsection{Le segment SCTP HEARTBEAT ACK}

\subsection{Le segment SCTP INIT}

\subsection{Le segment SCTP INIT ACK}

\subsection{Le segment SCTP SACK}

\subsection{Le segment SCTP SHUTDOWN}

\subsection{Le segment SCTP SHUTDOWN ACK}

\subsection{Le segment SCTP SHUTDOWN ACK COMPLETE}

%
\section{Le routage orienté destination de TCP/IP}

\section{Et ensuite ?}

% --------------------------------------------------------- Chapitre 3
\chapter{Introduction au filtrage IP}

\section{Qu'est-ce qu'un filtre IP ?}

\section{Termes et expressions du filtrage IP}

\section{Comment orrganiser un filtrage IP ?}

\section{Et ensuite ?}

% --------------------------------------------------------- Chapitre 4
\chapter{Introduction aux traductions d'adresse réseau (NATs)}

\section{À quoi sert un NAT et définitions}

\section{Avertissement d'utilisation du NAT}

\section{Exemple d'un NAT théorique}

\subsection{Ce qu'il faut pour construire un NAT}

\subsection{Positionnement d'un NAT}

\subsection{Positionnement d'un proxy}

\subsection{État final du NAT}

%
\section{Et ensuite ?}

% --------------------------------------------------------- Chapitre 5
\chapter{Préparatifs}

\section{Où obtenir iptables ?}

\section{Paramétrage du noyau}

\section{Paramétrage de l'espace utilisateur}

\subsection{Compiler les applications en espace utilisateur}

\subsection{Installation sur RedHat 7.1}

\section{Et ensuite ?}

% --------------------------------------------------------- Chapitre 6
\chapter{Traversée des tables et chaînes}

\section{Généralités}

\section{La table Mangle}

\section{La atble NAT}

\section{La table Raw}

\section{La table Filter}

\section{Chaînes définies par l'utilisateur}

\section{Et ensuite ?}

% --------------------------------------------------------- Chapitre 7
\chapter{La machine à états}

\section{Introduction}

\section{Les entrées de conntrack}

\section{Les états en espace utilisateur}

\section{Connections TCP}

\section{Connections UDP}

\section{Connections ICMP}

\section{Connections par défaut}

\section{Connections non suivies et table Raw}

\section{Protocoles complexes et trackers de connections}

\section{Et ensuite ?}

% --------------------------------------------------------- Chapitre 8
\chapter{Sauvegarder et restaurer un grand nombre de règles}

\section{Considération sur la vitesse}

\section{Inconvénients de la restauration}

\section{iptables-save}

\section{iptables-restore}

\section{Et ensuite ?}

% --------------------------------------------------------- Chapitre 9
\chapter{Construction de règle}

\section{Bases de la commande iptables}

\section{Tables}

\section{Commandes}

\section{Et ensuite ?}

% --------------------------------------------------------- Chapitre 10

\chapter{Correspondance avec iptables}

\section{Correspondance génériques}

\section{Correspondances implicites}

\subsection{Correspondance TCP}

\subsection{Correspondance UDP}

\subsection{Correspondance ICMP}

\subsection{Correspondances SCTP}

\section{Correspondances Explicites}




 
\end{document}
